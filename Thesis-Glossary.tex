%%%%%%%%%%%%%%%%%%%%%%%%%%%%%%%%%%%%%%%%%%%%%%%%%%%%%%%%%%%%%%%%%
%%                                                       GLOSSARY
%%%%%%%%%%%%%%%%%%%%%%%%%%%%%%%%%%%%%%%%%%%%%%%%%%%%%%%%%%%%%%%%%

% Reference using the schema
%
%   \ac{}   -- general: abbreviation except first time is full
%   \acs{}  -- short: abbreviation
%   \acl{}  -- long: definition
%   \acf{}  -- full: definition (abbreviation)
%
%   pluralization       -- append "p", e.g. \acp{}
%   capitalize first    -- capitalize first of command, e.g. \Ac{}
%   capitalize all      -- capitalize all of command, e.g. \AC{}
%
%   See page 85 of the documentation:
%   http://ctan.math.utah.edu/ctan/tex-archive/macros/latex/contrib/glossaries-extra/glossaries-extra-manual.pdf

\makeglossaries

% Credit to https://www.overleaf.com/learn/latex/Glossaries for the example

\newglossaryentry{latex}{
    name=latex,
    description={Is a mark up language specially suited for scientific documents}
}

\newglossaryentry{maths}{
    name=mathematics,
    description={Mathematics is what mathematicians do}
}

\newglossaryentry{formula}{
    name=formula,
    description={A mathematical expression}
}

\newabbreviation{gcd}{GCD}{Greatest Common Divisor}
\newabbreviation{lcm}{LCM}{Least Common Multiple}
